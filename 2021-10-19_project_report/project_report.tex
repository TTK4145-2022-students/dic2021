%%%%%%%%%%%%%%%%%%%%%%%%%%%%%%%%%%%%%%%%%%%%%%%%%%%%%%%%%%%%%%%%%%%%%%
%%        Copyright (c) 2021 Carsten Wulff Software, Norway
%% %%%%%%%%%%%%%%%%%%%%%%%%%%%%%%%%%%%%%%%%%%%%%%%%%%%%%%%%%%%%%%%%%%%
%% Created       : wulff at 2021-6-13
%% %%%%%%%%%%%%%%%%%%%%%%%%%%%%%%%%%%%%%%%%%%%%%%%%%%%%%%%%%%%%%%%%%%%
%%  The MIT License (MIT)
%%
%%  Permission is hereby granted, free of charge, to any person obtaining a copy
%%  of this software and associated documentation files (the "Software"), to deal
%%  in the Software without restriction, including without limitation the rights
%%  to use, copy, modify, merge, publish, distribute, sublicense, and/or sell
%%  copies of the Software, and to permit persons to whom the Software is
%%  furnished to do so, subject to the following conditions:
%%
%%  The above copyright notice and this permission notice shall be included in all
%%  copies or substantial portions of the Software.
%%
%%  THE SOFTWARE IS PROVIDED "AS IS", WITHOUT WARRANTY OF ANY KIND, EXPRESS OR
%%  IMPLIED, INCLUDING BUT NOT LIMITED TO THE WARRANTIES OF MERCHANTABILITY,
%%  FITNESS FOR A PARTICULAR PURPOSE AND NONINFRINGEMENT. IN NO EVENT SHALL THE
%%  AUTHORS OR COPYRIGHT HOLDERS BE LIABLE FOR ANY CLAIM, DAMAGES OR OTHER
%%  LIABILITY, WHETHER IN AN ACTION OF CONTRACT, TORT OR OTHERWISE, ARISING FROM,
%%  OUT OF OR IN CONNECTION WITH THE SOFTWARE OR THE USE OR OTHER DEALINGS IN THE
%%  SOFTWARE.
%%
%%%%%%%%%%%%%%%%%%%%%%%%%%%%%%%%%%%%%%%%%%%%%%%%%%%%%%%%%%%%%%%%%%%%%%

\documentclass[paper,10pt,a4paper]{IEEEtran}
\usepackage[final]{graphicx}
\ifCLASSINFOpdf
\graphicspath{{.}}
\DeclareGraphicsExtensions{.pdf}
\else
\graphicspath{{.}}
\DeclareGraphicsExtensions{.eps}
\fi
\usepackage{wrapfig}
\usepackage{upgreek}
\usepackage{amssymb,amsmath}
\usepackage{cite}
\usepackage{verbatim}
\usepackage{listings}
\usepackage{xcolor}
\usepackage{flafter}
\usepackage{booktabs}
\usepackage{url}
\usepackage{hyperref}
\usepackage{textcomp}
\usepackage{dirtytalk}
\definecolor{armygreen}{rgb}{0, 0.5, 0}
\newcommand{\missing}[1]{{ #1}}
\newcommand{\edit}[1]{{ #1}}
\newcommand{\secedit}[1]{{ #1}}
\newcommand{\deleted}[1]{{ #1}}
\newcommand{\moved}[1]{{ #1}}
%\usepackage{tabularx}
\usepackage{tabulary}
\newcommand{\req}[1]{Equation \ref{eqn:#1} }
\newcommand{\eqn}[1]{
  \begin{equation}
    #1
  \end{equation}}

\newcommand{\eqnr}[2]{
  \begin{equation}
    #1
    \label{eqn:#2}
  \end{equation}}


\newcommand{\qt}[1]{
  \begin{quote}
    \small
    \textit{
      #1
     }
   \end{quote}}


 \graphicspath{ {./} }

% -----------------------------------------------------------------
% Header
% -----------------------------------------------------------------
\begin{document}
\bibliographystyle{IEEEtran}
\title{What a project report should contain}
\author{Carsten~Wulff, \textit{2021-10-24}, v0.1.0 }
\maketitle

% -----------------------------------------------------------------
% Abstract
% -----------------------------------------------------------------
\begin{abstract}
  The abstract captures the interest of the reader, and explains in future tense
  what the reader will read. Key parameters, such as current consumption, area
  or gate count, are highlighted.
\end{abstract}

% -----------------------------------------------------------------
\section{Introduction}
% -----------------------------------------------------------------

The purpose of the introduction is to put the reader into the right frame of
mind. Introduce the problem statement, key references, the key contribution of
your work, and an outline of the
work presented. Think of the introduction
as explaining the ``Why'' of the work.

Although everyone has the same assignment for the project, you have chosen to
solve the problem in different ways. Explain what you consider the problem
statement, and tailor the problem statement to what the reader will read.

Key references, like \cite{klein01}, is introduced. Don't copy the paper text, write why they designed the circuit, how they chose to implement it, and
what they achieved. The reason we reference other papers in the introduction is
to show that we understand the current state-of-the-art. As such, maybe find
other, more recent, image sensors. Provide a summary where
state-of-the-art has moved since the original paper.

The outline should be included towards the end of the
introduction. The purpose of the outline is to make this document easy to read. A reader should
never be surprised by the text. All concepts should be eased into. We don't want
the reader to feel like they been thrown in at the end of a long story. As such,
if you chosen to solve the problem statement in a way not previously solved in a
key references, then you should explain that.

A checklist for all chapters can be seen in \ref{tbl:check}.

% ----------------------------------------------------------------
\section{Theory}
% -----------------------------------------------------------------
It is safe to assume that all readers have read the key reference
\cite{klein01}, if they have not, then expect them to do so. The purpose of the
theory section is not to demonstrate that you have read the paper, but rather,
highlight theory that the reader probably does not know. The theory section
should give sufficient explanation to bridge the gap between references, and
what you apply in this text.

% -----------------------------------------------------------------
\section{Implementation}
% -----------------------------------------------------------------
The purpose of the implementation is to explain what you did. How have you
chosen to architect the solution, how did you split it up in analog and digital
parts? Use one subsection per circuit.

For the analog, explain the design decisions you made, how did you pick the transistor sizes,
and the currents. Use clear figures (i.e. circuitikz),
don't use pictures from schematic editors. How does the circuit work? Did you make other choices than in
\cite{klein01}?

For the digital, how did you divide up the digital? What were the design choices you made? How
did you split it up into finite state machines and pixel array control? How did
you implement readout of the data? Explain what you did, and how it works. Use
state diagrams and block diagrams.

\section{Result}
The purpose of the results is to convince the reader that what you made
actually works. To do that, explain testbenches and simulation results.
The key to good results is to be critical of your own work. Do not try to oversell
the results. Your result should speak for them self.

For analog circuits, show results from each block. Highlight key
parameters, like current and delay of comparator. Demonstrate that the full analog system works. Show that the correct
digital value is stored in memory. Check memory value for multiple pixel voltages, either by
changing the exposure time, or changing the pixel current.

Show simulations that demonstrate that the digital works. Show how you read out the data from the pixel array.

\section{Discussion}
Explain what the circuit and results show. Be critical.

\section{Future work}
Give some insight into what is missing in the work. What should be the next steps?

\section{Conclusion}
Summarize why, how, what and what the results show.

\section{Appendix}
Include in appendix the necessary files to reproduce the work. One good way to
do it is to make a github repository with the files, and give a link here.

The SPICE and SystemVerilog for the actual circuit should be included into the
abstract directly. Testbenches, makefiles etc, can be linked via github.

\begin{table*}[thb]
  \centering
  \label{tbl:check}
\caption{Project report checklist}
\renewcommand{\arraystretch}{1.3}
  \small
\begin{tabulary}{\textwidth}{ |L|L|L|}
  \hline
  \textbf{Item} &  \textbf{Description} & \textbf{OK}\\
  \hline
  \multicolumn{3}{|l|}{\textbf{Introduction}} \\

  \hline
  Is the problem description clearly defined? &  Describe which parts of the
                                                  problem you chose to focus on.
                                                  The problem description should
                                                  match the results you've
                                                  achieved. & \\
  \hline
  Is there a clear explanation why the problem is worth solving?  & The reader
                                                                   might need
                                                                   help to
                                                                   understand
                                                                   why the
                                                                   problem is
                                                                   interesting &
  \\
  \hline
  Is status of state-of-the-art clearly explained? &  You should make sure that
                                                       you know what others have
                                                       done for the same
                                                       problem. Check
                                                       IEEEXplore. Provide
                                                       summary and references.
                                                       Explain how your problem
                                                       or solution is different
                                                       &
  \\
  \hline
  Is the key contribution clearly explained? & Highlight what you've achieved.
  What was your contribution? & \\
  \hline
  Is there an outline of the report? & Give a short summary of what the reader
                                       is about to read & \\
  \hline

  \multicolumn{3}{|l|}{\textbf{Theory}} \\
  \hline
  Is it possible for a reader skilled in the art to understand the work? & Have
                                                                          you
                                                                          included
                                                                          references
                                                                          to
                                                                          relevant
                                                                          papers
                      & \\
  \hline
  Is the theory section too long & The theory section should be less than 10 \%
                                   of the work & \\
  \hline
  \multicolumn{3}{|l|}{\textbf{Implementation}} \\
  \hline
  Are all circuits explained? & Have you explained how every single block works?
                      & \\
  \hline
  Are figures clear? & Remember to explain all colors, and all symbols. Explain
                       what the reader should understand from the figure. All
                       figures must be referenced in the text.& \\
  \hline
  \multicolumn{3}{|l|}{\textbf{Results}} \\
  \hline
  Is it clear how you verified the circuit? & It's a good idea to explain what
                                              type of testbenches you used. For
                                              example, did you use dc, ac or
                                              transient to verify your circuit?&
  \\
  \hline
  Are key parameters simulated? & You at least need current from VDD. Think
                                  through what you would need to simulate to
                                  prove that the circuit works. & \\
  \hline
  Have you tried to make the circuit fail? & Knowing how circuits fail will
                                            increase confidence that it will
                                            work under normal conditions. & \\
  \hline
  \multicolumn{3}{|l|}{\textbf{Discussion}} \\
  \hline
  Have you been critical of your own results? & Try to look at the verification
                                               from different perspectives. Play
                                               devil's advocate, try to think
                                               through what could go wrong, then
                                        explain how your verification proves that
                                        the circuit does work.& \\
  \hline
  \multicolumn{3}{|l|}{\textbf{Future work}} \\
  \hline
  Have you explained the next steps? & Imagine that someone reads your work.
                                       Maybe they want to reproduce it, and take
                                       one step further. What should that step
                                       be? & \\
  \hline
  \multicolumn{3}{|l|}{\textbf{Conclusion}} \\
  \hline
  No new information in conclusion. & Never put new information into conclusion.
                                      It's a summary of what's been done & \\
  \hline
  \multicolumn{3}{|l|}{\textbf{General comments}}\\
  \hline
  Story & Does the work tell a story, is it readable? Don't surprise the reader
          by introducing new topics without background information. & \\
  \hline
  Chronology & Don't let the report follow the timeline of the work done. What I mean
              by that is don't write ``first I did this, then I spent huge
              amount of time on this, then I did that''. No one cares what the
              timeline was. The report does not need to follow the same timeline
              as the actual work. & \\
  \hline
  Too much time & How much time you spent on something should not be
                  correlated to how much text there is in the report. No one
                  cares how much time you spent on something. The report is
                  about why, how, what and does it work. & \\
  \hline
  Length & A report should be concise. Only include what is necessary, but no more. Shorter is
           almost always better than longer. & \\
  \hline
  Template & Use
             \url{https://github.com/wulffern/dic2021/tree/main/2021-10-19_project_report}
             template for the report. Write in \LaTeX. You will need \LaTeX\  for your project and master thesis. Use \url{http://overleaf.com}
             if you're uncomfortable with local text editors and \LaTeX. & \\
 \hline
  Spellcheck & Always use a spellchecker. Miss-spelled words are
               annoying, and may change content and context (peaked versus
               piqued) & \\
 \hline

\end{tabulary}

\end{table*}


\begin{thebibliography}{1}
  \providecommand{\url}[1]{#1}

  \bibitem{klein01}
  Kleinfelder, Lim, Liu, Gamal ``A 10 000 Frames/s CMOS Digital Pixel Sensor'',
  JSSC, VOL 36, NO 12, 2001




\end{thebibliography}






\end{document}
