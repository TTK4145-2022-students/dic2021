%%%%%%%%%%%%%%%%%%%%%%%%%%%%%%%%%%%%%%%%%%%%%%%%%%%%%%%%%%%%%%%%%%%%%%
%%        Copyright (c) 2021 Carsten Wulff Software, Norway
%% %%%%%%%%%%%%%%%%%%%%%%%%%%%%%%%%%%%%%%%%%%%%%%%%%%%%%%%%%%%%%%%%%%%
%% Created       : wulff at 2021-6-13
%% %%%%%%%%%%%%%%%%%%%%%%%%%%%%%%%%%%%%%%%%%%%%%%%%%%%%%%%%%%%%%%%%%%%
%%  The MIT License (MIT)
%%
%%  Permission is hereby granted, free of charge, to any person obtaining a copy
%%  of this software and associated documentation files (the "Software"), to deal
%%  in the Software without restriction, including without limitation the rights
%%  to use, copy, modify, merge, publish, distribute, sublicense, and/or sell
%%  copies of the Software, and to permit persons to whom the Software is
%%  furnished to do so, subject to the following conditions:
%%
%%  The above copyright notice and this permission notice shall be included in all
%%  copies or substantial portions of the Software.
%%
%%  THE SOFTWARE IS PROVIDED "AS IS", WITHOUT WARRANTY OF ANY KIND, EXPRESS OR
%%  IMPLIED, INCLUDING BUT NOT LIMITED TO THE WARRANTIES OF MERCHANTABILITY,
%%  FITNESS FOR A PARTICULAR PURPOSE AND NONINFRINGEMENT. IN NO EVENT SHALL THE
%%  AUTHORS OR COPYRIGHT HOLDERS BE LIABLE FOR ANY CLAIM, DAMAGES OR OTHER
%%  LIABILITY, WHETHER IN AN ACTION OF CONTRACT, TORT OR OTHERWISE, ARISING FROM,
%%  OUT OF OR IN CONNECTION WITH THE SOFTWARE OR THE USE OR OTHER DEALINGS IN THE
%%  SOFTWARE.
%%
%%%%%%%%%%%%%%%%%%%%%%%%%%%%%%%%%%%%%%%%%%%%%%%%%%%%%%%%%%%%%%%%%%%%%%

\documentclass[technote,10pt,a4paper]{IEEEtran}
\usepackage[final]{graphicx}
\ifCLASSINFOpdf
\graphicspath{{.}}
\DeclareGraphicsExtensions{.pdf}
\else
\graphicspath{{.}}
\DeclareGraphicsExtensions{.eps}
\fi
\usepackage{wrapfig}
\usepackage{upgreek}
\usepackage{amssymb,amsmath}
\usepackage{cite}
\usepackage{verbatim}
\usepackage{listings}
\usepackage{xcolor}
\usepackage{flafter}
\usepackage{booktabs}
\usepackage{url}
\usepackage{textcomp}
\usepackage{dirtytalk}
\definecolor{armygreen}{rgb}{0, 0.5, 0}
\newcommand{\missing}[1]{{ #1}}
\newcommand{\edit}[1]{{ #1}}
\newcommand{\secedit}[1]{{ #1}}
\newcommand{\deleted}[1]{{ #1}}
\newcommand{\moved}[1]{{ #1}}
\newcommand{\req}[1]{Equation \ref{eqn:#1} }
\newcommand{\eqn}[1]{
  \begin{equation}
    #1
  \end{equation}}

\newcommand{\eqnr}[2]{
  \begin{equation}
    #1
    \label{eqn:#2}
  \end{equation}}


\newcommand{\qt}[1]{
  \begin{quote}
    \small
    \textit{
      #1
     }
   \end{quote}}


 \graphicspath{ {./} }

% -----------------------------------------------------------------
% Header
% -----------------------------------------------------------------
\begin{document}
\bibliographystyle{IEEEtran}
\title{Diodes}
\author{Carsten~Wulff, \textit{2021-07-08}, v0.1.0 }
\maketitle

% -----------------------------------------------------------------
% Abstract
% -----------------------------------------------------------------
\begin{abstract}
I explain how diodes work.
\end{abstract}

% -----------------------------------------------------------------
\section{Why}
% -----------------------------------------------------------------
Diodes are a magical \footnote{It doesn't stop being magic just because you know
  how it works -- Terry Pratchett, The Wee Free Men} semiconductor device that mostly conducts current in one
direction. This is a useful feature, and it is one way to convert from an AC
voltage to a DC voltage using a
\url{https://en.wikipedia.org/wiki/Diode_bridge}.

In integrated circuits we don't intentionally use them that much. They have a forward voltage
of about 0.5 V, and for low voltage circuits they are not that useful. But there
are a few instances where they are very useful.

All integrated circuits are plagued by electrostatic discharge (ESD), both during
assembling printed circuit boards (PCB), and after, when people touch the PCB.
The ESD events can push huge currents into our intregrated circuits -- 2 kV ESD
zap is approximately 1.3 A --, and normal transistor simply don't survive those
currents. At the pins of the IC we often use diodes to carry the ESD current
safely between the pins of the device, and the grounded pin.

Although we don't intentionally use them, they are an inherent feature of almost
all MOSFETs. The drain and source regions of an NMOS are doped with a donor element, and
the bulk is doped with acceptors. This formes diodes between drain/source and
bulk. These parasitic diodes have an capacitance that loads all circuits, and
must be taken into account.

Another useful feature of the diode is the expoential relationship betwen the
forward current, and the voltage across the device. If you push a constant
current into a diode, then small changes to the current does not change the
diode voltage significantly. You could use this as a reference voltage, however,
the forward current does change with the temperature, so it requires sligthly
more than one diode and a current to make a reference voltage that is stable over temperature.

% -----------------------------------------------------------------
\section{How}
% -----------------------------------------------------------------
Silicon is a crystal where all electrons are used in covalent bonds between the
silicon atoms. If we ignore temperature, then none of the electronics are
free to move. The temperature, the vibrations of the atoms, do sometimes
break the covalent bond, so there is a continuous generation of electron/hole
pairs in pure silicon.

To figure out the intrinsic carrier concentration we need
to delve deep into the solid state physics (see intrinsic.py). The intrinsic
carrier concentration is a function of the Fermi energy, the bandgap, the mass
of carriers, a bunch of constants, and temperature.

At room temperature this intrinsic carrier consentration is about
$ n_{i} =  1 \times 10^{16} carriers/m^3$.

That may sound like a big number, however, if we calculate the electrons per
$um^{3}$ it's
$n_{i} = \frac{1 \times 10^{16}}{(1 \times 10^{6})^{3}} carriers/\mu m^{3}< 1  $,
so there are really not that many free carriers in intrinsic silicon.

We can change the property of silicon by introducing other elements. Phosphor
has one more electron/proton than silicon, Boron has one less electron/proton.
Injecting these elements into the silicon crystal lattice changes the number of
free electron/holes (those not used in covalent bonds), this is commonly
referred to as doping. If we have an element
with more electrons we call it a donor, and the donor concentration $N_{D}$.
Since the crystal now has an abundance of electrons we call it n-type.
If the element has less electrons we call it an acceptor, and the acceptor
concentration $N_{A}$. Since the crystal now has an abundance of holes, we
call it p-type. Usually these doping concentrations are larger than the
intrinsic carrier concentration, from maybe $10^{21}$ to $10^{27}$
$carriers/m^{3}$. To separate between these concentrations we use $p-,p,p+$ or
$n-, n, n+$. In most instances the doping concentration is so much higher that
the intrinsic carrier concentration that we can safely assume that the number of
electroncs, and number of holes are the same as $N_{D}$ and $N_{A}$.

In a p-type material, although holes dominate, there will still be a minority of
electrons moving around, which is given by
\eqn{p_{n} = \frac{n_{i}^{2}}{N_{D}}}
, and a similar equation for the hole concentration in an n-type.


In a p-type crystal there is a majority of holes, and a minority of electrons.
Thus we name holes majority carriers, and electrons minority carriers. For
n-type it's opposite.

\section{What}
Imagine an n-type material, and a p-type material, both are neutral in charge,
because they have the same number of electrons and protons. The free carriers
will move around the material constantly.

Now imagine we bring the two materials together. Some of the electrons in the
n-type will wander over to the p-type material, and visa versa. Here they will
find an opposite charge, and will get locked in place. They will become stuck. This creates a depletion region with immobile charges.
Where as the two materials used to be neutrally charged, there will now be a
build up of negative charge on the p-side, and positive charge on the n-side.
 There will also be a
field created by the charge difference, and a built-in voltage will develop
across the depletion region. The built in voltage depends on the carrier
concentrations, and is given by \req{bv}, where $k$ is Boltzmanns constant, $q$ is
the charge of an electron.

\eqnr{ \Phi_0 = \frac{kT}{q} ln\left(  \frac{N_A N_D}{n_i^2} \right)}{bv}

\section{Reverse bias}
We can apply an external voltage to the pn diode. If we apply a field in the
same direction as the built-in voltage, we call it reverse bias. Under reverse
bias the current in the diode is small.

As mentioned before, we continuously
have electron/hole pairs generated by the temperature. In addition, we can have
electron/hole pairs generated by for example photons (photo diodes), or impact ionization
(charges at high speed, like radiation). Those electron/hole pairs come into
existence in the depletion region, or happen to wander into the depletion region
before recombining will be swept across the depletion region due to the electric
field.  The
electron will drift to the n-type, and holes will drift to the p-type. This
drift creates a leakage current in the diode.

To estimate the leakage current we would need to know how many electron/hole
pairs are generated per second, and how many reach the depletion region before
recombining. Not at all a trivial calculation. However, we do expect that the
leakage current also doubles per $11^{o}C$, similar to the $n_{i}$.

The width of a depletion region in the n-type material can be approximated by
\req{width}, where $l_{1}$ is \req{l1},
where $\varepsilon_{0}$ is the permittivity of free space
($8.854 \times 10^{12} $ F/m), and $K_{s} = 11.8$ the relative
permittivity of silicon. At $V_{R} = 0$, the depletion width is $l_{1}$. As we
increase $V_{R}$ the depletion region will grow, but it's not proportional to
$V_{R}$. The depletion width has the same equation for  p-type, just
replace $N_{A}$ with $N_{D}$.

\eqnr{ x_n(V_R) = l_1 \sqrt{1 + \frac{V_R}{\Phi_0}} }{width}

\eqnr{ l_1 = \sqrt{\frac{2K_s\varepsilon_0}{q\Phi_0}\frac{N_A}{N_D(N_A +
      N_D)}} }{l1}

Remember that $I = C \frac{dV}{dt}$, and $I = \frac{dQ}{dt}$ thus $ C= \frac{dQ}{dV}$, so if we find the charge in the
depletion region, then we can calculate the small signal capacitance. For the
n-side the depletion region
charge per unit area can be approximated by $Q = qN_D x_n(V_R)$ so the
capacitance per unit area is \req{dcap}, where $C_{j0}$ is \req{cj0}
\eqnr{ C = \frac{C_{j0}}{\sqrt{1 + \frac{V_R}{\Phi_0} } }}{dcap}
\eqnr{ C_{j0} = \frac{qN_D l_1}{2} =
  \sqrt{\frac{qK_s\varepsilon_0}{2\Phi_0}\frac{N_A N_D}{N_A +
      N_D}} }{cj0}

% -----------------------------------------------------------------
\section{Avalance breakdown}
% -----------------------------------------------------------------
If the reverse bias across the depletion region becomes large enough, then any
minority carrier that stumbles into the depletion region can become
accelerated to high enough energy such that it creates a electron/hole pair (ehp) when
it impacts the crystal lattice (impact ionization), this new electron will also
be accelerated, impact, new ehp, and so on. This avalanche of electrons will
give a current that can effectively increase without limit. Suddenly you have a
enormous current flowing in your reverse diode. This avalance current can be triggered
by ESD, and it has killed devices, which make pretty pictures in a scanning
electron microscope, but it's never what we want.



% -----------------------------------------------------------------
\section{What}
% -----------------------------------------------------------------

% -----------------------------------------------------------------
\section{Conclusion}
% -----------------------------------------------------------------



\begin{thebibliography}{1}
  \providecommand{\url}[1]{#1}

  \bibitem{bbs47}
  Brittain, Bardeen, Shockley ``Point-contact transistor'', online:\url{https://en.wikipedia.org/wiki/Transistor}




\end{thebibliography}






\end{document}
